Dans cette partie, sont décrits les algorithmes de principe associés aux fonctions écrites en langage C, ainsi qu'un lexique concernant les variables intermédiaires des fonctions.

Les lexiques des variables d'entrée, sortie et entrée/sortie sont disponibles dans le code source directement.

\section{Gestion de la pile}
  La gestion de la pile s'effectue grâce aux fichiers \texttt{stack.c} et \texttt{stack.h}.
  
  \subsection{init}
      \begin{algo}[informal] %principe
      \ALGO{init}
      \BEGIN
        \STATE[Il n'y a aucune erreur]{On initialise le code de retour à 1}
        \STATE{On initialise la taille max de la pile}
        \STATE[Pour indiquer qu'elle est vide]{On initialise l'indice du haut de la pile à -1}
        \STATE{On alloue l'espace de la pile}
        \IF{l'allocation n'est pas réussie}
          \STATE{Le code d'erreur passe à 0}
        \ENDIF
        \RETURN{code d'erreur}
      \END
    \end{algo}

    \begin{algo}[informal] %lexique
      \VAR
        \DECLVAR{ret}{code d'erreur, 0 si il y a une erreur de création de la pile, 1 sinon}
      \ENDVAR
    \end{algo}

  \subsection{supp}
    Ici, nous libérons simplement de tableau de valeurs de la pile, puisque celui-ci est alloué dynamiquement lors de la création.
  
  \subsection{empty}
    Cette fonction teste simplement si l'indice du haut de la pile est -1, ce qui veut dire qu'il n'y a aucun élément dans la pile. Ainsi la valeur 1 sera retournée. Sinon la valeur 0 est retournée.
  
  \subsection{full}
    Cette fonction vérifie si l'indice de l'élément en haut de la pile est égal à la taille max de la pile (moins 1, car les tableaux commencent à 0). Si c'est le cas, on renvoie 1 pour signaler que la pile est pleine, et 0 sinon.
  
  \subsection{pop}
    \begin{algo}[informal] %principe
    \ALGO{pop}
    \BEGIN
      \STATE[On n'a pas dépilé]{On initialise le code de retour à 0}
      \IF{la pile n'est pas vide}
        \STATE{On dépile l'élément dans une variable en Input/Output}
        \STATE[On a dépilé et récupéré l'élément]{Le code de retour passe à 1}
        \STATE[On retranche 1 à l'indice précédent]{On modifie l'indice de l'élément en haut de la pile}
      \ENDIF
      \RETURN{code de retour}
    \END
  \end{algo}

  \begin{algo}[informal] %lexique
    \VAR
      \DECLVAR{ok}{code de retour, 0 si on n'a pas dépilé, 1 si on a dépilé la valeur en haut de la pile}
    \ENDVAR
  \end{algo}
  
  \subsection{top}
    \begin{algo}[informal] %principe
    \ALGO{top}
    \BEGIN
      \STATE[Pas d'élément lu]{On initialise le code de retour à 0}
      \IF{la pile n'est pas vide}
        \STATE{On copie la tête de pile dans une variable en I/O}
        \STATE[On a dépilé]{Le code d'erreur passe à 1}
      \ENDIF
      \RETURN{code de retour}
    \END
  \end{algo}

  \begin{algo}[informal] %lexique
    \VAR
      \DECLVAR{ok}{code de retour, 0 si on n'a pas récupéré l'élément en haut de la pile, 1 si on l'a récupéré}
    \ENDVAR
  \end{algo}
  
  \subsection{push}
    \begin{algo}[informal] %principe
    \ALGO{push}
    \BEGIN
      \STATE[L'élément n'est pas ajouté dans la pile]{On initialise le code de retour à 0}
      \IF{la pile n'est pas pleine}
        \STATE{On incrémente l'indice de l'élément en haut de la pile}
        \STATE{On place l'élément dans le tableau de la pile, à l'indice précédement modifié}
        \STATE[L'élément est empilé]{Le code de retour passe à 1}
      \ENDIF
      \RETURN{code de retour}
    \END
  \end{algo}

  \begin{algo}[informal] %lexique
    \VAR
      \DECLVAR{ok}{code de retour, 0 si on n'a pas empilé, 1 si on a empilé la valeur}
    \ENDVAR
  \end{algo}

\section{Dérécursivation de la fonction}
  La fonction récursive ainsi que sa version itérative se trouvent dans les fichiers \texttt{truc.c} et \texttt{truc.h}.
  \subsection{TRUC}
    Cette fonction étant l'énoncé du TP, nous ne détaillerons ainsi ni le principe ni les variables utilisées dans cet algorithme.
    
    \documentclass{standalone}
\usepackage[usenames,dvipsnames]{xcolor}
\usepackage[utf8]{inputenc}
\usepackage[T1]{fontenc}
\usepackage[frenchb]{babel}
\usepackage{amsmath}
\usepackage{tikz}
\usetikzlibrary{shapes,arrows}

\begin{document}
  \begin{tikzpicture}
    % style des noeuds
    \tikzstyle{debutfin}=[ellipse,draw,fill=Yellow!75]
    \tikzstyle{instruct}=[rectangle,draw,fill=Orange!75,text width=2.5cm,align=center]
    \tikzstyle{test}=[diamond, aspect=2.5,thick,
    draw=ForestGreen,fill=LimeGreen!75]
    \tikzstyle{es}=[rectangle,draw,rounded corners=4pt,fill=RoyalBlue!50]
    % style des flèches
    \tikzstyle{suite}=[thick,->,>=stealth']
    % placement des noeuds
    \node[debutfin] (debut) at (-2,6) {Début};
    \node[instruct] (init) at (-2,4) {$sl := s;$ $il := i;$ $p := \mathrm{INIT}();$};
    \node[test] (test1) at (-2,2) {$sl = 0 ?$};
    \node[test] (test2) at (-2,0) {$sl < 0 \text{ ou } il < N ?$};
    \node[instruct] (push) at (-2,-2.5) {$\mathrm{Empiler}(sl);$ $\mathrm{Empiler}(il);$};
    \node[instruct] (inc_double) at (-2,-5) {$sl := sl - P(il);$ $il := il + 1;$};
    \node[instruct] (r_true) at (2,2) {$R := Vrai;$};
    \node[instruct] (r_false) at (2,0) {$R := Faux;$};
    \node[test] (test3) at (6,2) {$\mathrm{Vide}(p)?$};
    \node[instruct] (pop) at (6,0) {$\mathrm{Depiler}(il);$ $\mathrm{Depiler}(sl);$};
    \node[test] (test4) at (6,-2) {$R = Vrai ?$};
    \node[es] (print) at (6,-4) {$\mathrm{Ecrire}(P(il));$};
    \node[instruct] (ilpp) at (10,-2) {$il := il + 1$};
    \node[instruct] (ret) at (10,2) {$\mathrm{Retourner} R;$};
    \node[debutfin] (fin) at (10,1) {Fin};
    % Placement des flèches
    \draw[suite] (debut) -- (init);
    \draw[suite] (init) -- (test1.north);
    \draw[suite] (test1.south) -- (test2.north) node[near start,right]{non};
    \draw[suite] (test2.south) -- (push.north) node[near start,right]{non};
    \draw[suite] (push.south) -- (inc_double.north);
    \draw[suite] (test1.east) -- (r_true) node[near start,above]{oui};
    \draw[suite] (test2.east) -- (r_false) node[near start,above]{oui};
    \draw[suite] (r_true) -- (test3.west);
    \draw[suite] (r_false.east) -| (4,0) |- (test3.west);
    \draw[suite] (test3.east) -- (ret) node[near start,above]{oui};
    \draw[suite] (ret.south) -- (fin.north);
    \draw[suite] (test3.south) -- (pop.north) node[near start,right]{non};
    \draw[suite] (pop.south) -- (test4.north);
    \draw[suite] (test4.east) -- (ilpp) node[near start,above]{non};
    \draw[suite] (test4.south) -- (print.north) node[near start,right]{oui};
    \draw[suite] (print.south) |- +(0,-0.5) -| (4,1);
    \draw[suite] (inc_double.south) |- +(0,-0.5) -| +(-2.5,0) |- (-2,3);
    \draw[suite] (ilpp.east) -| +(0.5,0) |- +(0,5) |- (-2,3);
    \draw[suite] (4,-3)--(4,-2);
  \end{tikzpicture}
\end{document}

    
  \subsection{truc\_iter}
    \begin{algo}[informal] %principe
      \ALGO{truc\_iter (Principe)}
      \BEGIN
        \STATE{Copie des paramètres d'entré dans des variables locales, sl et il}
        \STATE{Initialisation de la pile de la même taille que le tableau statique}
        \IF{l'initialisation a réussi}
          \REPEAT
            \WHILE{sl \GT 0 \AND il \LE N}
              \STATE{On envoie il dans la pile}
              \STATE{sl \AFF sl - P(il)}
              \STATE{On incrémente il}
            \ENDWHILE
            \IF{sl \EQ 0}
              \STATE{Le booléen de retour est à Vrai}
              \WHILE{la pile n'est pas vide}
                \STATE{On récupère il à partir de la pile}
                \STATE{On ajoute P(il) à sl}
                \STATE{On affiche P(il)}
              \ENDWHILE
            \ELSE
              \STATE{Le booléen de retour est à Faux}
              \IF{la pile n'est pas vide}
                \STATE{On récupère il à partir de la pile}
                \STATE{On ajoute P(il) à sl}
                \STATE{On incrémente il}
              \ENDIF
            \ENDIF
          \ENDREPEAT[while]{la pile n'est pas vide}
          \STATE{On supprime la pile}
        \ENDIF
        \RETURN{Booléen de retour}
      \END
    \end{algo}

    \begin{algo}[informal] %lexique
      \VAR
        \DECLVAR{sl}{copie locale du nombre s passé en paramètre. Représente le nombre à décomposer}
        \DECLVAR{il}{copie locale du nombre i passé en paramètre. Représente le nombre d'entiers du tableau à utiliser pour décomposer s}
        \DECLVAR{r}{booléen de retour, indique 1 si on a obtenue la somme s, 0 sinon}
        \DECLVAR{p}{pile}
        \DECLVAR{P}{tableau d'entiers, défini statiquement}
        \DECLVAR{N}{taille du tableau P}
      \ENDVAR
    \end{algo}
