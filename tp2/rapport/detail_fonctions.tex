Dans cette partie, sont décrits les algorithmes de principe associés aux fonctions écrites en langage C, ainsi qu'un lexique concernant les variables intermédiaires des fonctions.

Le lexique des variable d'entrée, sortie et entrée/sortie sont disponibles dans le code source directement.

\section{Gestion des news}
  La gestion des news s'effectue grace aux fichiers \texttt{gestion\_news.c} et \texttt{gestion\_news.h}.
  
  \subsection{charger}
    \begin{algo}[informal]
      \ALGO{Charger (Principe)}
      \BEGIN
        \STATE{On ouvre en lecture seule le fichier dont le nom est passé en paramètre}
        \STATE[erreur d'ouverture de fichier]{On initialise le code de retour à 1}
        \IF{on a pu l'ouvrir}
          \STATE[pas de problème d'ouverture de fichier]{Le code de retour passe à 0}
          \WHILE{l'on n'arrive pas à la fin du fichier}
            \STATE{On lit une ligne}
            \STATE{On stocke les date de début et de fin de message dans des variables}
            \STATE{On stocke le message dans une chaîne de caractère}
            \STATE{On crée une nouvelle cellule à partir des variables ci-dessus récupérées}
            \IF{on a pu créer la cellule}
              \STATE{On l'insère dans la liste chaînée}
            \ELSE
              \STATE[erreur allocation de cellule]{Le code de retour passe a 2}
            \ENDIF
          \ENDWHILE
          \STATE{On ferme le fichier}
        \ENDIF
        \RETURN{le code d'erreur}
      \END
    \end{algo}
    \begin{algo}[informal]
      \VAR
        \DECLVAR{ret}{entier qui retourne un code d'erreur 
          \begin{minipage}{0.5\linewidth}
            \begin{itemize}[noitemsep,nolistsep]
              \item 0 si aucune erreur\phantom{p}
              \item 1 si problème lors de l'ouverture du fichier
              \item 2 si problème lors de la création de la cellule
            \end{itemize}
          \end{minipage}
        }
        \DECLVAR{*fichier}{descripteur de fichier, ouvert a partir du nom passé en paramètres}
        \DECLVAR{debut, fin}{entiers servant à stocker les dates de début et de fin de message lors du traitement du fichier}
        \DECLVAR{*scan}{chaîne de caractère permettant de stocker le message lors du traitement du fichier}
        \DECLVAR{*tmp}{pointeur de cellule qui servira de cellule temporaire à chaîner dans la liste}
      \ENDVAR
    \end{algo}

  \subsection{sauver}
    \begin{algo}[informal]
      \ALGO{Sauver (Principe)}
      \BEGIN
        \STATE{On crée un nouveau fichier à partir du nom donné en paramètre}
        \STATE[erreur de création de fichier]{On initialise le code de retour à 1}
        \IF{l'on a pu créer le fichier}
          \WHILE{l'on n'est pas à la fin de la liste}
            \STATE{On écrit les élément de la cellule en cours de traitement dans le fichier}
            \STATE{On passe à la cellule suivante}
          \ENDWHILE
          \STATE{On ferme le fichier}
          \STATE[pas d'erreur]{On passe le code de retour à 0}
        \ENDIF
        \RETURN{le code d'erreur}
      \END
      \vspace{0.3cm}
      \VAR
        \DECLVAR{ret}{entier représentant le code d'erreur de la fonction \begin{minipage}{0.45\textwidth}
        \begin{itemize}[noitemsep,nolistsep]
          \item 0 aucune erreur\phantom{p}
          \item 1 erreur de création ou d'ouverture du fichier
        \end{itemize} \end{minipage}}
        \DECLVAR{*cour}{pointeur sur la cellule en cours de traitement dans la liste chaînée}
        \DECLVAR{*fichier}{descripteur de fichier crée à partir du nom passé en paramètre}
      \ENDVAR
    \end{algo}

  \subsection{getDate}
    Cette fonction est fournie pour le TP, nous ne la détaillerons donc pas.
  
  \subsection{afficher\_message}
    \begin{algo}[informal]
      \ALGO{afficher\_message (Principe)}
      \BEGIN
        \STATE{On écrit les données d'une cellule en récupérant ses attributs}
      \END
    \end{algo}
  
  \subsection{afficher\_liste}
    \begin{algo}[informal]
      \ALGO{afficher\_liste (Principe)}
      \BEGIN
        \STATE{On initialise l'élément courant au début de la liste}
        \WHILE{l'élément courant n'est pas NULL}
          \STATE{On affiche les élément de la cellule avec \textit{afficher\_message}}
          \STATE{On passe à la cellule suivante}
        \ENDWHILE
      \END
    \end{algo}
    \begin{algo}[informal]
      \VAR
        \DECLVAR{*cour}{pointeur sur la cellule en cours de traitement dans la liste chaînée}
      \ENDVAR
    \end{algo}

  \subsection{afficher\_messages\_date}
    \begin{algo}[informal]
      \ALGO{afficher\_messages\_date (Principe)}
      \BEGIN
        \STATE{On initialise l'élément courant au début de la liste}
        \WHILE{la date de début de l'élément courant est supérieure à la date à traiter}
          \STATE{On passe à la cellule suivante}
        \ENDWHILE
        \WHILE{la liste n'est pas finie}
          \WHILE{la date de fin de l'élément courant est ultérieure à la date recherchée}
            \STATE{On affiche le message de l'élément courant}
            \STATE{On passe à la cellule suivante}
          \ENDWHILE
          \IF{la liste n'est pas finie}
            \STATE{On passe à la cellule suivante}
          \ENDIF
        \ENDWHILE
      \END
    \end{algo}

    \begin{algo}[informal]
      \VAR
        \DECLVAR{*cour}{pointeur sur la cellule en cours de traitement dans la liste chaînée}
      \ENDVAR
    \end{algo}

  \subsection{afficher\_messages\_jour}
    \begin{algo}[informal]
      \ALGO{afficher\_messages\_jour (Principe)}
      \BEGIN
        \STATE{On appelle la fonction \textit{afficher\_messages\_date} avec en paramètre la date du jour obtenue avec \textit{getDate}}
      \END
    \end{algo}
    
        \begin{algo}[informal]
      \VAR
        \DECLVAR{Aucune variable intermédiaire n'est utilisée dans cette fonction}{}
      \ENDVAR
    \end{algo}
          
  \subsection{afficher\_messages\_motif}
    \begin{algo}[informal] %principe
      \ALGO{afficher\_messages\_motif (Principe)}
      \BEGIN
        \STATE{On initialise l'élément courant au début de la liste}
        \WHILE{l'élément courant n'est pas NULL}
          \IF{le message contenu dans la cellule courant contient le motif}
            \STATE{On affiche ce message}
          \ENDIF
          \STATE{On passe à l'élément suivant}
        \ENDWHILE
      \END
    \end{algo}

    \begin{algo}[informal] %lexique
      \VAR
        \DECLVAR{*cour}{Pointeur sur la cellule en cours de traitement dans la liste chaînée}
      \ENDVAR
    \end{algo}

  \subsection{supprimer\_obsoletes}
    \begin{algo}[informal] %principe
      \ALGO{supprimer\_obsoletes (Principe)}
      \BEGIN
        \STATE{On récupère la date du jour dans un entier}
        \STATE{On initialise un pointeur double **prec au début de la liste (ceci afin d'économiser l'utilisation du'une variable supplémentaire)}
        \WHILE{*prec n'est pas NULL}
          \IF{la date de fin de message de *prec est inférieure à la date du jour}
            \STATE{On supprime cette cellule à partir du précédent prec}
          \ELSE
            \STATE{On passe à l'élément suivant}
          \ENDIF
        \ENDWHILE
      \END
    \end{algo}

    \begin{algo}[informal] %lexique
      \VAR
        \DECLVAR{date}{Date du jour récupérée avec la fonction \textit{getDate}}
        \DECLVAR{**prec}{Pointeur double servant à parcourir la liste en ayant toujours accès à l'élément précédent d'une cellule}
      \ENDVAR
    \end{algo}

  \subsection{remplacer\_date}
    \begin{algo}[informal]
      \ALGO{remplacer\_date (Principe)}
      \BEGIN
        \STATE{On initialise l'élément courant que début de la liste}
        \WHILE{l'élémént courant n'est pas NULL \ANDTHEN La date de début de message de l'élément courant est supérieure à la date à modifier}
          \STATE{On passe à l'élément suivant}
        \ENDWHILE
        \WHILE{l'élément courant n'est pas NULL \ANDTHEN La date de début de message de l'élément courant est égale à la date à modifier}
          \IF{la date de fin de l'élément courant est supérieure ou égale à la nouvelle date}
            \STATE{On modifie la date de début de message de l'élément courant à la nouvelle date passée en paramètre}
          \ENDIF
          \STATE{On passe à l'élément suivant}
        \ENDWHILE
        \STATE{On sauvegarde la liste courante dans un fichier temporaire}
        \STATE{On libère la liste}
        \STATE{On recharge la liste à partir du fichier temporaire, ce qui va recréer la liste triée}
      \END
    \end{algo}

    \begin{algo}[informal] %lexique
      \VAR
        \DECLVAR{*cour}{Pointeur sur la cellule en cours de traitement dans la liste chaînée}
      \ENDVAR
    \end{algo}

\section{Gestion de la liste chaînée}
  \subsection{adj\_cell}
    \begin{algo}[informal] %principe
      \ALGO{adj\_cell (Principe)}
      \BEGIN
        \STATE{On a **prec un pointeur sur l'adresse de l'élément suivant à l'élément après lequel insérer}
        \STATE{L'élément suivant de l'élément à ajouter est *prec, l'élément suivant de son précédent}
        \STATE{L'élément suivant du précédent devient l'élément à ajouter}
      \END
    \end{algo}

    \begin{algo}[informal] %lexique
      \VAR
        \DECLVAR{Aucune variable intermédiaire n'est utilisée dans cette fonction}{}
      \ENDVAR
    \end{algo}

  \subsection{rech\_prec}
    \begin{algo}[informal] %principe
      \ALGO{rech\_prec (Principe)}
      \BEGIN
        \STATE{On initialise **prec, un pointeur sur l'adresse du début de la liste}
        \WHILE{*prec n'est pas NULL \ANDTHEN la date de début de *prec est supérieure à la date après laquelle insérer}
          \STATE{prec prend l'adresse de l'adresse de l'élément suivant dans la liste}
        \ENDWHILE
        \STATE{La variable d'entrée/sortie \textit{existe} prend la valeur 1 si la date cherchée est présente, 0 sinon}
        \RETURN{prec}
      \END
    \end{algo}

    \begin{algo}[informal] %lexique
      \VAR
        \DECLVAR{Aucune variable intermédiaire n'est utilisée dans cette fonction}{}
      \ENDVAR
    \end{algo}

  \subsection{supp\_cell}
    \begin{algo}[informal] %principe
      \ALGO{supp\_cell (Principe)}
      \BEGIN
        \STATE{On sauvegarde l'adresse de l'élément à supprimer dans un pointeur temporaire}
        \STATE{L'élément suivant du précédent devient l'élément suivant de l'élément à supprimer}
        \STATE{On libère le texte de la cellule à supprimer (car celui-ci est alloué dynamiquement)}
        \STATE{On libère la cellule à supprimer}
      \END
    \end{algo}

    \begin{algo}[informal] %lexique
      \VAR
        \DECLVAR{*elt}{pointeur temporaire sur la cellule à supprimer}
      \ENDVAR
    \end{algo}
  
  \subsection{liberer\_liste}
    \begin{algo}[informal] %principe
      \ALGO{liberer\_liste (Principe)}
      \BEGIN
        \WHILE{la liste n'est pas vide}
          \STATE{On supprime les cellules une à une}
        \ENDWHILE
        \STATE{On place le pointeur de tête de la liste à NULL}
      \END
    \end{algo}

    \begin{algo}[informal] %lexique
      \VAR
        \DECLVAR{Aucune variable intermédiaire n'est utilisée dans cette fonction}{}
      \ENDVAR
    \end{algo}

  \subsection{ins\_cell}
    \begin{algo}[informal] %principe
      \ALGO{ins\_cell (Principe)}
      \BEGIN
        \STATE{On récupère un pointeur sur l'adresse de l'élément avant lequel insérer avec la fonction \textit{rech\_prec}}
        \STATE{On ajoute l'élément dans la liste avec la fonction \textit{adj\_cell} en utilisant les variables prec et le pointeur sur l'élément à ajouter}
      \END
    \end{algo}

    \begin{algo}[informal] %lexique
      \VAR
        \DECLVAR{existe}{booléen de présence de l'élément recherché, modifié dans \textit{rech\_prec}}
        \DECLVAR{**prec}{pointeur sur l'adresse de l'élément après lequel insérer}
      \ENDVAR
    \end{algo}

  \subsection{creer\_cell}
    \begin{algo}[informal] %principe
      \ALGO{ (Principe)}
      \BEGIN
        \STATE{On alloue un élément de la liste chaînée}
        \IF{l'allocation a réussi}
          \STATE{On place les dates de début et de fin de message dans les champs appropriés}
          \STATE{On alloue l'espace nécessaire pour le message dans le champ texte}
          \STATE{On recopie le texte passé en entrée dans le champ texte de la cellule}
        \ENDIF
        \RETURN{le nouvel élément créé}
      \END
    \end{algo}

    \begin{algo}[informal] %lexique
      \VAR
        \DECLVAR{*elt}{pointeur sur le nouvel élément créé, ce sera également le retour de la fonction. Si celui-ci est à NULL, cela signifie qu'il y a eu un problème lors de son allocation}
      \ENDVAR
    \end{algo}
    
