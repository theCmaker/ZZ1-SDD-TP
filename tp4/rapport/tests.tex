\subsection{Matrice nulle}
	Fichier d'entrée (le fichier de matrice nulle est vide):
	\inputminted[frame=single,label=test\_nulle]{text}{../tests/test_nulle}
	Résultat:
	\inputminted[frame=single,label=Résultat matrice nulle]{text}{../tests/resultat_test_nulle}
\subsection{Matrice quelconque}
	Fichier d'entrée:
	\inputminted[frame=single,label=test\_1]{text}{../tests/test_1}
	Fichier matrice:
	\inputminted[frame=single,label=matrice\_1]{text}{../tests/matrice_1}
	Résultat:
	\inputminted[frame=single,label=Résultat matrice quelconque,tabsize=4]{text}{../tests/resultat_test_1}
\subsection{Matrice ligne}
	Fichier d'entrée:
	\inputminted[frame=single,label=test\_2]{text}{../tests/test_2}
	Fichier matrice:
	\inputminted[frame=single,label=matrice\_2]{text}{../tests/matrice_2}
	Résultat:
	\inputminted[frame=single,label=Résultat matrice ligne,tabsize=2,breaklines=true]{text}{../tests/resultat_test_2}
\newpage
\subsection{Matrice très creuse (un seul élément non-nul)}
	\label{subs:mat_creuse}
	Fichier d'entrée:
	\inputminted[frame=single,label=test\_3]{text}{../tests/test_3}
	Fichier matrice:
	\inputminted[frame=single,label=matrice\_3]{text}{../tests/matrice_3}
	Résultat:
	\inputminted[frame=single,label=Résultat matrice très creuse,tabsize=2,breaklines=true]{text}{../tests/resultat_test_3}
\newpage
\subsection{Bonne utilisation de la mémoire}
	Pour vérifier la bonne libération de la mémoire, nous avons utilisé \texttt{valgrind} avec le programme et un fichier de test. Aucun bloc de mémoire n'est perdu, le retour texte de valgrind est présenté en figure \ref{fig:valgrind}.
	\begin{figure}[H]
	  \inputminted[frame=single,label=Resultat Valgrind]{text}{../tests/valgrind_test_1}
	  \caption{Passage dans l'outil \texttt{valgrind}}
	  \label{fig:valgrind}
	\end{figure}

	Notons que l'utilisation de cette structure de données est en effet intéressante pour les matrices creuses:
	
	Par exemple pour le test de la matrice très creuse (page \pageref{subs:mat_creuse}), \texttt{valgrind} nous indique que le programme a utilisé au total 1968 octets. Le seul stockage de cette matrice sous forme de tableau aurait nécessité $4 \times 20 \times 24 = 1920$ octets. 

	L'inconvénient de cette structure est que la table majeure occupe 8 à 12 octets par bloc de ligne, et que le nombre de ces lignes est fixé (ici 50+1, soit environ 400 octets mobilisés au minimum), on peut donc estimer qu'il faut près d'une centaine d'éléments nuls au minimum pour rendre cette structure de données intéressante du point de vue de la mémoire. 

	On pourrait éventuellement proposer d'autres tailles pour la table majeure, ou bien une longueur dynamique (mais au détriment des performances lorsqu'il s'agira de redimensionner l'espace, i.e. lors de l'insertion).
