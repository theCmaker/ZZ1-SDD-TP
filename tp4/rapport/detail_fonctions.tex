Dans cette partie, sont décrits les algorithmes de principe associés aux fonctions écrites en langage C, ainsi qu'un lexique concernant les variables intermédiaires des fonctions.

Les lexiques des variables d'entrée, sortie et entrée/sortie sont disponibles dans le code source directement.

\section{Gestion des listes chaînées}
  Les fonctions de gestion des listes chaînées peuvent être trouvées dans les fichiers \texttt{list.c} et \texttt{list.h}.
  Les algorithmes de principe de ces fonctions ont déjà été fournis dans le \texttt{TP 1}. Ils ne seront donc pas inclus ici.

\section{Gestion de la matrice}
  La gestion de la matrice s'effectue avec les fonctions contenues dans \texttt{mat.c} et définies dans \texttt{mat.h}.
    
  \newpage
\subsection{lire\_matrice}
  \begin{algo}[informal] %principe
    \ALGO{ (Principe)}
    \BEGIN
      \STATE{Ouverture du fichier en lecture seule}
      \STATE{Initialisation du code d'erreur à 1}
      \STATE{Initialisation de la matrice}
      \IF{le fichier est correctement ouvert \AND la matrice est initialisée}
        \WHILE{code d'erreur \EQ 1 \AND on a pas lu tout le fichier}
          \STATE{On récupère les valeurs de la ligne du fichier}
          \STATE{On insère la valeur à la bonne place dans la matrice}
          \IF{l'insertion n'a pas réussie}
            \STATE{Le code d'erreur passe à 0}
          \ENDIF
        \ENDWHILE
      \ELSE
        \STATE{Le code d'erreur passe à 0}
      \ENDIF
      \RETURN{Code d'erreur}
    \END
  \end{algo}
  
  \begin{algo}[informal] %lexique
    \VAR
      \DECLVAR{*f}{descripteur sur le fichier contenant les informations sur la matrice}
      \DECLVAR{row}{entier pour récupérer le numéro de la ligne dans le fichier}
      \DECLVAR{col}{entier pour récupérer le numéro de la colonne dans le fichier}
      \DECLVAR{val}{valeur à insérer dans la matrice}
      \DECLVAR{res}{code d'erreur (1 aucune erreur, 0 sinon)}
    \ENDVAR
  \end{algo}
  
\subsection{init\_mat}
  \begin{algo}[informal] %principe
    \ALGO{init\_mat (Principe)}
    \BEGIN
      \STATE{Initialisation du code d'erreur à 1}
      \STATE{Initialise le nombre de lignes de la matrice à 0}
      \STATE{Allocation du tableau de lignes de la matrice}
      \IF{le tableau de ligne est alloué}
        \STATE{On place une valeur de ligne maximale pour la première ligne}[Ceci permet d'avoir un bon résultat lors de la première recherche dichotomique]
        \STATE{Le code d'erreur passe à 1}
      \ENDIF
      \RETURN{Code d'erreur}
    \END
  \end{algo}
  
  \begin{algo}[informal] %lexique
    \VAR
      \DECLVAR{res}{code d'erreur (1 si aucun problème, 0 sinon)}
    \ENDVAR
  \end{algo}
  
\subsection{rech\_dich}
  \begin{algo}[informal] %principe
    \ALGO{rech\_dich (Principe)}
    \BEGIN
      \STATE{Initialisation d'un entier deb à 0}
      \STATE{Initialisation d'un entier fin au nombre de lignes de la matrice}
      \IF{ligne cherchée \LT numéro de la dernière ligne de la matrice}
        \WHILE{deb \NEQ fin}
          \STATE{mil prend la valeur (deb + fin) / 2}
          \IF{ligne cherchée \LT numéro de la ligne à la place mil dans la matrice}
            \STATE{fin prend la valeur mil}
          \ELSE
            \STATE{deb prend mil + 1}
          \ENDIF
        \ENDWHILE
      \ENDIF
      \STATE{Modification du booléen existe}
      \RETURN{deb}
    \END
  \end{algo}
  
  \begin{algo}[informal] %lexique
    \VAR
      \DECLVAR{deb}{entier représentant la position de départ de la recherche dichotomique, c'est également le retour de la fonction indiquant où se trouve l'élément cherché ou son adresse d'insertion}
      \DECLVAR{fin}{entier représentant la position de fin de la recherche dichotomique}
      \DECLVAR{mil}{entier représentant le milieu entre deb et fin}
      \DECLVAR{*existe}{adresse du booleen indiquant si la ligne cherchée se trouve ou non déjà dans la matrice (1 si elle existe, 0 sinon)}
    \ENDVAR
  \end{algo}
  
  
