Dans cette partie, sont décrits les algorithmes de principe associés aux fonctions écrites en langage C, ainsi qu'un lexique concernant les variables intermédiaires des fonctions.

Les lexiques des variables d'entrée, sortie et entrée/sortie sont disponibles dans le code source directement.

\section{Gestion de la pile}
  La gestion de la pile s'effectue grâce aux fichiers \texttt{stack.c} et \texttt{stack.h}.
  Les algorithmes de principe des différentes fonctions ont été précédement détaillés dans le \texttt{tp 2}, nous ne les détaillerons donc pas.

\section{Gestion des listes chaînées}
  Les fonctions de gestion des listes chaînées peuvent être trouvées dans les fichiers \texttt{list.c} et \texttt{list.h}.
  Les algorithmes de principe de ces fonctions ont également été fourni dans le \texttt{tp 1}. Ils ne seront donc pas inclus ici.

\section{Gestion de l'arbre}
  La gestion de l'arbre s'effectue avec les fonctions contenues dans \texttt{tree.c} et \texttt{tree.h}.
    
  \newpage
  \subsection{creerArbre}
    \begin{algo}[informal] %principe
      \ALGO{creerArbre (Principe)}
      \BEGIN
        \RETURN{}
      \END
    \end{algo}

    \begin{algo}[informal] %lexique
      \VAR
        \DECLVAR{}{}
      \ENDVAR
    \end{algo}

  \subsection{insererMot}
    \begin{algo}[informal] %principe
      \ALGO{insererMot (Principe)}
      \BEGIN
        \RETURN{}
      \END
    \end{algo}

    \begin{algo}[informal] %lexique
      \VAR
        \DECLVAR{}{}
      \ENDVAR
    \end{algo}

  \subsection{afficherArbre}
    \begin{algo}[informal] %principe
      \ALGO{afficherArbre (Principe)}
      \BEGIN
        \RETURN{}
      \END
    \end{algo}

    \begin{algo}[informal] %lexique
      \VAR
        \DECLVAR{}{}
      \ENDVAR
    \end{algo}
    
  \subsection{afficherPoint}
    \begin{algo}[informal] %principe
      \ALGO{afficherPoint (Principe)}
      \BEGIN
        \RETURN{}
      \END
    \end{algo}

    \begin{algo}[informal] %lexique
      \VAR
        \DECLVAR{}{}
      \ENDVAR
    \end{algo}

  \subsection{libererArbre}
    \begin{algo}[informal] %principe
      \ALGO{libererArbre (Principe)}
      \BEGIN
        \RETURN{}
      \END
    \end{algo}

    \begin{algo}[informal] %lexique
      \VAR
        \DECLVAR{}{}
      \ENDVAR
    \end{algo}
