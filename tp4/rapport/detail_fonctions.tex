Dans cette partie, sont décrits les algorithmes de principe associés aux fonctions écrites en langage C, ainsi qu'un lexique concernant les variables intermédiaires des fonctions.

Les lexiques des variables d'entrée, sortie et entrée/sortie sont disponibles dans le code source directement.

\section{Gestion des listes chaînées}
  Les fonctions de gestion des listes chaînées peuvent être trouvées dans les fichiers \texttt{list.c} et \texttt{list.h}.
  Les algorithmes de principe de ces fonctions ont déjà été fournis dans le \texttt{TP 1}. Ils ne seront donc pas inclus ici.

\section{Gestion de la matrice}
  La gestion de la matrice s'effectue avec les fonctions contenues dans \texttt{mat.c} et définies dans \texttt{mat.h}.
    
  \newpage
\subsection{lire\_matrice}
  \begin{algo}[informal] %principe
    \ALGO{ (Principe)}
    \BEGIN
      \STATE{Ouverture du fichier en lecture seule}
      \STATE{Initialisation du code d'erreur à 1}
      \STATE{Initialisation de la matrice}
      \IF{le fichier est correctement ouvert \AND la matrice est initialisée}
        \WHILE{code d'erreur \EQ 1 \AND on a pas lu tout le fichier}
          \STATE{On récupère les valeurs de la ligne du fichier}
          \STATE{On insère la valeur à la bonne place dans la matrice}
          \IF{l'insertion n'a pas réussie}
            \STATE{Le code d'erreur passe à 0}
          \ENDIF
        \ENDWHILE
      \ELSE
        \STATE{Le code d'erreur passe à 0}
      \ENDIF
      \RETURN{Code d'erreur}
    \END
  \end{algo}
  
  \begin{algo}[informal] %lexique
    \VAR
      \DECLVAR{*f}{descripteur sur le fichier contenant les informations sur la matrice}
      \DECLVAR{row}{entier pour récupérer le numéro de la ligne dans le fichier}
      \DECLVAR{col}{entier pour récupérer le numéro de la colonne dans le fichier}
      \DECLVAR{val}{valeur à insérer dans la matrice}
      \DECLVAR{res}{code d'erreur (1 aucune erreur, 0 sinon)}
    \ENDVAR
  \end{algo}
  
\subsection{init\_mat}
  \begin{algo}[informal] %principe
    \ALGO{init\_mat (Principe)}
    \BEGIN
      \STATE{Initialisation du code d'erreur à 1}
      \STATE{Initialise le nombre de lignes de la matrice à 0}
      \STATE{Allocation du tableau de lignes de la matrice}
      \IF{le tableau de ligne est alloué}
        \STATE{On place une valeur de ligne maximale pour la première ligne}[Ceci permet d'avoir un bon résultat lors de la première recherche dichotomique]
        \STATE{Le code d'erreur passe à 1}
      \ENDIF
      \RETURN{Code d'erreur}
    \END
  \end{algo}
  
  \begin{algo}[informal] %lexique
    \VAR
      \DECLVAR{res}{code d'erreur (1 si aucun problème, 0 sinon)}
    \ENDVAR
  \end{algo}
  
\subsection{rech\_dich}
  \begin{algo}[informal] %principe
    \ALGO{rech\_dich (Principe)}
    \BEGIN
      \STATE{Initialisation d'un entier deb à 0}
      \STATE{Initialisation d'un entier fin au nombre de lignes de la matrice}
      \IF{ligne cherchée \LT numéro de la dernière ligne de la matrice}
        \WHILE{deb \NEQ fin}
          \STATE{mil prend la valeur (deb + fin) / 2}
          \IF{ligne cherchée \LT numéro de la ligne à la place mil dans la matrice}
            \STATE{fin prend la valeur mil}
          \ELSE
            \STATE{deb prend mil + 1}
          \ENDIF
        \ENDWHILE
      \ENDIF
      \STATE{Modification du booléen existe}
      \RETURN{deb}
    \END
  \end{algo}
  
  \begin{algo}[informal] %lexique
    \VAR
      \DECLVAR{deb}{entier représentant la position de départ de la recherche dichotomique, c'est également le retour de la fonction indiquant où se trouve l'élément cherché ou son adresse d'insertion}
      \DECLVAR{fin}{entier représentant la position de fin de la recherche dichotomique}
      \DECLVAR{mil}{entier représentant le milieu entre deb et fin}
      \DECLVAR{*existe}{adresse du booleen indiquant si la ligne cherchée se trouve ou non déjà dans la matrice (1 si elle existe, 0 sinon)}
    \ENDVAR
  \end{algo}
  
\subsection{decal\_rows}
  Cette procédure décale simplement les éléments de la table principale d'une case vers la droite jusqu'à la case spécifiée en paramètre ; en partant de la dernière case.

\subsection{inser\_row}
  Cette procédure va dans un premier temps effectuer un décalage des éléments de la table vers la droite. Ceci va permettre d'avoir une place pour la nouvelle ligne.
  On incrément le nombre de lignes de la matrice, on insère le numéro de ligne dans la table principale et on place le pointeur vers la liste chaînée représentant les colonnes de cette ligne sur NULL.

\subsection{inser\_val}
  \begin{algo}[informal] %principe
    \ALGO{inser\_val (Principe)}
    \BEGIN
      \STATE{Initialisation du code d'erreur à 1}
      \STATE{Recherche la ligne d'insertion par Dichotomie}
      \IF{la ligne n'existe pas}
        \STATE{On l'insère avec la fonction inser\_val}
      \ENDIF
      \STATE{On crée la nouvelle cellule}
      \IF{la cellule est correctement crée}
        \STATE{On ajoute la cellule dans la liste chaînée des colonnes, sur la ligne correspondante}
        \IF{l'indice de la colonne ajoutée \LT maximum de colonne courant}
          \STATE{On actualise le maximum de colonnes}
        \ENDIF
      \ELSE
        \STATE{Le code d'erreur passe à 0}
      \ENDIF
      \RETURN{Code d'erreur}
    \END
  \end{algo}
  
  \begin{algo}[informal] %lexique
    \VAR
      \DECLVAR{existe}{booleen indiquant si la ligne cherchée existe déjà ou non}
      \DECLVAR{res}{code d'erreur (1 si aucun problème, 0 sinon)}
      \DECLVAR{*c}{cellule crée, qui sera insérée dans la matrice}
    \ENDVAR
  \end{algo}
  
\subsection{element}
  \begin{algo}[informal] %principe
    \ALGO{element (Principe)}
    \BEGIN
      \STATE{Initialisation de la valeur de l'élément à 0}
      \IF{les indices sont dans la matrice}
        \STATE{Recherche par dichotomie de la ligne}
        \IF{la ligne cherchée existe}
          \STATE{Recherche de la colonne dans la liste chaînée des colonnes}
          \IF{la colonne existe}
            \STATE{La valeur de l'élément prend la valeur de la cellule qui est trouvée}
          \ENDIF
        \ENDIF
      \ENDIF
      \RETURN{valeur de l'élément}
    \END
  \end{algo}
  
  \begin{algo}[informal] %lexique
    \VAR
      \DECLVAR{elt}{valeur de l'élément qui sera renvoyée}
      \DECLVAR{**prec}{pointeur sur l'élément trouvé dans la recherche de la colonne}
      \DECLVAR{r}{entier indiquant la position de la ligne cherchée dans la table majeure}
      \DECLVAR{existe}{booleen indiquant si la ligne cherchée existe ou non}
    \ENDVAR
  \end{algo}
  
\subsection{afficher\_matrice}
  \begin{algo}[informal] %principe
    \ALGO{afficher\_matrice (Principe)}
    \BEGIN
      \STATE{Initialise la ligne courante ligne\_cour à 0}
      \FORGEN{pour chaque ligne de la matrice Faire}
        \STATE{L'indice j de colonne est à 1}
        \IF{la ligne courante est présente dans la matrice}
          \STATE{Stocke l'adresse de la première colonne col\_cour de la ligne courante}
          \WHILE{col\_cour \NEQ NULL \AND j \LE nombre maximum de colonnes}
            \IF{col\_cour \EQ j}
              \STATE{Affiche la valeur de la cellule}
              \STATE{col\_cour passe sur la cellule suivante}
            \ELSE
              \STATE{Affiche 0}
            \ENDIF
            \STATE{Incrémente j}
          \ENDWHILE
          \STATE{Incrémentation de la ligne}
        \ENDIF
        \WHILE{on est pas au bout de la ligne courante}
          \STATE{On complète avec des 0}
          \STATE{Incrémente j}
        \ENDWHILE
        \STATE{Afficher un retour à la ligne}
      \ENDFOR
    \END
  \end{algo}
  
  \begin{algo}[informal] %lexique
    \VAR
      \DECLVAR{i}{indice de ligne}
      \DECLVAR{j}{indice de colonne}
      \DECLVAR{ligne\_cour}{indice de la ligne courante, existante dans la matrice}
      \DECLVAR{*col\_cour}{pointeur parcourant les cellules des colonnes des lignes de la matrice}
      \DECLVAR{max\_col}{nombre de colonnes maximales dans la matrice}
      \DECLVAR{max\_row}{nombre de lignes maximales dans la matrice}
    \ENDVAR
  \end{algo}

\subsection{liberer\_matrice}
  Cette fonction permet de libérér la mémoire occupée par la matrice. On fait une boucle for sur chaque élément de la table majeure et dans cette boucle on libère chaque listes avec la fonction \textit{liberer\_liste} (permettant de libérer une liste chaînée).
  Une fois sortis de la boucle, on libère le tableau de la table majeure et on indique que la nombre maximal de lignes et de colonnes et de 0.
