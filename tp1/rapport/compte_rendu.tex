\documentclass{report}
\usepackage[utf8]{inputenc} %encodage entrée
\usepackage[T1]{fontenc}
\usepackage{graphicx} %images
\usepackage[usenames,svgnames]{xcolor} %couleurs
%\usepackage{cmbright} %fonte
\usepackage[light,math]{iwona}
\usepackage{tikz} %schémas
\usepackage{algo} %mise en forme d'algos
\usepackage{minted} %mise en forme de code source
\usepackage{framed} %cadres et bordures
\usepackage[frenchb]{babel} %langue
\usepackage{amsmath} %symboles maths
\usepackage{subcaption} %légendes et sous-figures
\usepackage{enumitem} %formatage des listes à puces
\usepackage[nobottomtitles]{titlesec} %formatage des chapitres
\usepackage{blindtext}
\usepackage[a4paper]{geometry} %mise en page
\usepackage[hidelinks]{hyperref}%liens

\input{settings/colors}
\input{settings/algo}
\input{settings/mintedC}
\input{settings/mintedMakefile}
\input{settings/mintedBash}
\input{settings/hyperrefSettings}

\hypersetup{
  pdftitle={TP 1 - Gestion de news à partir d'une liste chaînée}
}

%styles et formatage
\geometry{scale=0.8,centering}
\frenchbsetup{StandardLists=true}
\newcommand{\hsp}{\hspace{20pt}}
\titleformat{\chapter}[hang]{\LARGE\bfseries}{\thechapter\hsp\textcolor{lightgray}{|}\hsp}{0pt}{\LARGE\bfseries}

\newcommand{\NIL}{\texttt{\small NIL}}
\usetikzlibrary{decorations.pathreplacing} %grandes accolades des schémas
\tikzset{%styles schémas
    case1/.style={rectangle,draw,fill=RoyalBlue!50,minimum size=7.5mm,anchor=base,text height=1.5ex,text depth=.25ex,text centered},
    case/.style={case1},
    case2/.style={rectangle,draw,fill=Red!50,minimum size=7.5mm,anchor=base,text height=1.5ex,text depth=.25ex,text centered},
    head/.style={case1, fill=yellow},
    dead/.style={case1, fill=none, pattern=north west lines},
    void/.style={minimum size=7.5mm},
    vers/.style={thick,->, >= latex},
    ptr/.style={case1, fill=gray!50}
}

\begin{document}
  \begin{titlepage}
  \newcommand{\HRule}{\rule{\linewidth}{0.5mm}}
  \center
  \null{}
  \vspace{3cm}

  \textsc{\LARGE ISIMA Première Année}\\[1.5cm]
  \textsc{\Large Compte-Rendu de TP}\\[0.5cm]
  \textsc{\LARGE Structures de Données}\\[1.5cm]
  \HRule \\[0.4cm]
  { \huge \bfseries Dérécursification à l'aide d'une pile}\\
  \HRule \\[1.5cm]

  \begin{minipage}{0.4\textwidth}
    \begin{flushleft} \large
      Benjamin BARBESANGE\\
      Pierre-Loup PISSAVY\\
      {\normalsize\textit{Groupe G21}}
    \end{flushleft}
  \end{minipage}
  ~
  \begin{minipage}{0.4\textwidth}
    \begin{flushright} \large
      \emph{Enseignant :} \\
      Michelle CHABROL
    \end{flushright}
  \end{minipage}\\[4cm]

  {\large mars 2015}\\[3cm]

  \vfill

  \includegraphics[width=6cm]{settings/ISIMA_logo.pdf}\\[1cm]
\end{titlepage}

  \setlength{\parskip}{10pt}
  \setlength{\parindent}{0pt}
  \chapter{Présentation}
    Le but de ce TP est de concevoir un ensemble de fonctions permettant de gérer des news sous forme de messages, chacun d'entre eux ayant une date de début et de fin de validité. On doit faire usage d'une liste chaînée. Les news sont ordonnées dans la liste chaînée selon l'ordre décroissant de la date de début (de la plus récente à la plus ancienne).

    Les messages et informations satellites sont enregistrés dans un fichier, à raison d'une ligne par message. Ce fichier est supposé correct.

    Les opérations suivantes sont permises:
    \begin{itemize}
      \item Charger une liste depuis un fichier,
      \item Sauvegarder une liste dans un fichier,
      \item Afficher les messages du jour,
      \item Supprimer les messages obsolètes,
      \item Modifier une date de début sur tous les messages,
      \item Afficher tous les messages contenant une chaîne particulière.
    \end{itemize}

    \section{Structure de données employée}
      \begin{figure}[h]
  \begin{subfigure}{0.5\textwidth}
    \subcaption{Structure utilisée}
    \label{fig:struct}
    \begin{center}
    \shorthandoff{:}
    \begin{tikzpicture}[scale=0.5]
      \matrix at (0,0) {
        \node [case1] (a1) {max}; & \node [case1] (a11) {top}; & \node [case1,ptr] (a12) {val}; \\[1cm]
      };
      \matrix at (0, -4){
        \node [case2] (b1) {Valeurs dans la pile};\\
      };
      \draw [vers] (a12.south) |- +(0,-1) -| (b1.north);
    
    \end{tikzpicture}
    \shorthandon{:}
    \end{center}
  \end{subfigure}
  \begin{subfigure}{0.5\textwidth}
    \subcaption{Code}
    \label{fig:struct_code}
    \begin{center}
      \cfile[firstline=6,lastline=10]{../src/stack.h}
    \end{center}
  \end{subfigure}
  \caption{Structure et code correspondant}
\end{figure}

    \section{Organisation du code source}
      Nous avons défini deux modules, le premier est dédié à la gestion des listes chaînées (adjonction, recherche, suppression etc.), et le second à la gestion des news (lecture, sauvegarde, modifications etc.). Enfin, le programme principal utilise conjointement ces deux modules pour réaliser le traitement voulu.
      \subsection{Gestion de la pile}
  \begin{itemize}
	\item \bashinline{src/stack.h}
	\item \bashinline{src/stack.c}
  \end{itemize}

\subsection{Gestion des listes chaînées}
  \begin{itemize}
    \item \bashinline{src/list.h}
    \item \bashinline{src/list.c}
  \end{itemize}

\subsection{Gestion de l'arbre}
  \begin{itemize}
    \item \bashinline{src/tree.h}
    \item \bashinline{src/tree.c}
  \end{itemize}
  
\subsection{Programme principal}
  \begin{itemize}
    \item \bashinline{src/main.c}
  \end{itemize}

  \chapter{Détails du programme}
    \section{Gestion de liste chaînée}
  \cfile{../src/stack.h}
  \pagebreak
  \cfile{../src/stack.c}

\section{Gestion de news}
  \cfile{../src/truc.h}
  \cfile{../src/truc.c}

\pagebreak
\section{Programme principal}
  \cfile{../src/main.c}


 \chapter{Principes et lexiques des fonctions}
    \section{Gestion des news}
      La gestion des news s'effectue grace aux fichiers \textbf{gestion\_news.c} et \textbf{gestion\_news.h}.
      
      \subsection{charger}
Principe :
On ouvre en lecture seule le fichier dont le nom est passé en paramètre
On initialise le code de retour à 1 [erreur d'ouverture de fichier]

Si on a pu l'ouvrir Alors
  Le code de retour passe à 0 [pas de problème d'ouverture de fichier]
  Tant qu'on arrive pas à la fin du fichier Faire
    On lit une ligne
    On stocke les date de début et de fin de message dans des variables
    On stocke le message dans une chaîne de caractère
    On crée une nouvelle cellule à partir des variables ci-dessus récupérées
    
    Si on a pu créer la cellule Alors
      On l'insère dans la liste chaînée
    Sinon
      Le code de retour passe a 2 [erreur allocation de cellule]
    Fin si
  Fin tant que
  On ferme le fichier
Fin si
On retourne le code d'erreur

        \subsubsection{Lexique}
          \begin{itemize}
            \item ret : entier qui retourne un code d'erreur
            \begin{itemize}
              \item 0 si aucune erreur
              \item 1 si problème lors de l'ouverture du fichier
              \item 2 si problème lors de la création de la cellule
            \end{itemize}
            \item *fichier : descripteur de fichier, ouvert a partir du nom passé en paramètres
            \item debut, fin : entiers servant à stocker les dates de début et de fin de message lors du traitement du fichier
            \item *scan : chaîne de caractère permettant de stocker le message lors du traitement du fichier
            \item *tmp : pointeur de cellule qui servira de cellule temporaire à chaîner dans la liste
          \end{itemize}
      
      \subsection{sauver}
Principe :
On crée un nouveau fichier à partir du nom donné en paramètre
On initialise le code de retour à 1 [erreur de création de fichier]

Si on a pu créer le fichier Alors
  Tant qu'on est pas à la fin de la liste Faire
    On écrit les élément de la cellule en cours de traitement dans le fichier
    On passe à la cellule suivante
  Fin tant que
  On ferme le fichier
  On passe le code de retour à 0 [pas d'erreur]
Fin si
On retourne le code d'erreur
        
        \subsubsection{Lexique}
        \begin{itemize}
          \item ret : entier représentant le code d'erreur de la fonction
          \begin{itemize}
            \item 0 : aucune erreur
            \item 1 : erreur de création ou d'ouverture du fichier
          \end{itemize}
          \item *cour : pointeur sur la cellule en cours de traitement dans la liste chaînée
          \item *fichier : descripteur de fichier crée à partir du nom passé en paramètre
        \end{itemize}

      \subsection{getDate}
Cette fonction est fournie pour le TP, nous ne la détaillerons donc pas.
      
      \subsection{afficher\_message}
Principe :
On écrit les données d'une cellule en récupérant ses attributs

        \subsubsection{Lexique}
          Aucune variable intermédiaire utilisée dans cette fonction.
      
      \subsection{afficher\_liste}
Principe :
On itilialise l'élément courant au début de la liste
Tant que l'élément courant n'est pas NULL Faire
  On affiche les élément de la cellule avec \textit{afficher\_message}
  On passe à la cellule suivante
Fin tant que

        \subsubsection{Lexique}
        \begin{itemize}
          \item *cour : pointeur sur la cellule en cours de traitement dans la liste chaînée
        \end{itemize}
      
        \subsection{traiter\_elt\_date}
Principe :
On initialise l'élément courant au début de la liste
Tant que la date de début de l'élément courant est supérieure à la date à traiter Faire
  On passe à la cellule suivante
Fin tant que

Tant que la date de début de l'élément courant est égale à la date recherchée
  On applique la fonction passée en paramètre à la cellule courante
  On passe à la cellule suivante
Fin tant que

        \subsubsection{Lexique}
        \begin{itemize}
          \item *cour : pointeur sur la cellule en cours de traitement dans la liste chaînée
        \end{itemize}

      \subsection{afficher\_messages\_date}
Principe :
On appelle la fonction \textit{traiter\_elt\_date} avec la date de debt des messages à afficher et l'adresse de la fonction \textit{afficher\_message}

        \subsubsection{Lexique}
          Aucune variable intermédiaire n'est utilisée dans cette fonction

      \subsection{afficher\_messages\_jour}
Principe :
On appelle la fonction \textit{afficher\_messages\_date} avec en paramètre la date du jour obtenue avec \textit{getDate}
      
      \subsection{afficher\_messages\_motif}
Principe :
        
        \subsubsection{Lexique}
        \begin{itemize}
          \item 
        \end{itemize}
      
      \subsection{supprimer\_obsoletes}
Principe :

        \subsubsection{Lexique}
        \begin{itemize}
          \item 
        \end{itemize}

      \subsection{remplacer\_date}
Principe :

        \subsubsection{Lexique}
        \begin{itemize}
          \item 
        \end{itemize}      

    \section{Gestion de la liste chaînée}

  \chapter{Compte rendu d'exécution}
    \section{Makefile}
      \makefilefile{../src/Makefile}

    \section{Jeux de tests}
      %Exécution du programme avec le fichier suivant:
\inputminted[frame=single,label=Test]{text}{../tests/test_insert}

\begin{figure}[H]
	\centering
	\includegraphics[width=18cm,clip=true,trim=1cm 4cm 5cm 12cm]{../tests/ddd_graph/chargement}
	\caption{Après lecture du fichier, contenu de \texttt{liste}}
\end{figure}

\begin{figure}[h!]
	\centering
	\includegraphics[width=15cm,clip=true,trim=1cm 4cm 5cm 14cm]{../tests/ddd_graph/nettoyage}
	\caption{Après suppression des messages obsolètes, contenu de \texttt{liste}}
\end{figure}

\begin{figure}[h!]
	\centering
	\includegraphics[width=15cm,clip=true,trim=1cm 4cm 5cm 14cm]{../tests/ddd_graph/modification}
	\caption{Après modification de date de début, contenu de \texttt{liste}}
\end{figure}

\begin{figure}[h!]
	\centering
	\includegraphics[width=2cm,clip=true,trim=1cm 4.5cm 25cm 15cm]{../tests/ddd_graph/suppression}
	\caption{Après suppression de la liste, contenu de \texttt{liste}}
\end{figure}

\begin{figure}[h!]
	\centering
	\inputminted[frame=single,label=Terminal]{text}{../tests/resultat_test_insert}
	\caption{Exécution du programme sur la sortie standard}
\end{figure}

\newpage

\begin{figure}[h!]
	\centering
	\inputminted[frame=single,label=Terminal]{text}{../tests/valgrind_report}
	\caption{Exécution avec valgrind}
\end{figure}

Voici un petit exemple de la recherche de motif sur le même fichier d'entrée. On recherche dans ce cas la chaîne \texttt{"liste"}, immédiatement après avoir chargé le fichier dans la liste.

\begin{figure}[h!]
	\centering
	\inputminted[frame=single,label=Terminal]{text}{../tests/resultat_test_motif}
	\caption{Exécution de la recherche de motif}
\end{figure}

\end{document}