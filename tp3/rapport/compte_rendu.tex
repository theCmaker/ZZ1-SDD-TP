\documentclass{report}
\usepackage[utf8]{inputenc} %encodage entrée
\usepackage[T1]{fontenc}
\usepackage{graphicx} %images
\usepackage[usenames,dvipsnames]{xcolor} %couleurs
\usepackage[light,math]{iwona}
\usepackage{tikz} %schémas
\usepackage{algo} %mise en forme d'algos
\usepackage{minted} %mise en forme de code source
\usepackage{framed} %cadres et bordures
\usepackage[frenchb]{babel} %langue
\usepackage{amsmath} %symboles maths
\usepackage{caption} %legendes
\usepackage{subcaption} %légendes et sous-figures
\usepackage{multicol}
\usepackage{enumitem} %formatage des listes à puces
\usepackage[nobottomtitles]{titlesec} %formatage des chapitres
\usepackage{blindtext}
\usepackage[a4paper]{geometry} %mise en page
\usepackage[hidelinks]{hyperref}%liens
\usepackage{rotating}

\input{settings/colors}
\input{settings/algo}
\input{settings/mintedC}
\input{settings/mintedMakefile}
\input{settings/mintedBash}
\input{settings/hyperrefSettings}
\input{settings/chainedListBis}

\tikzset{%styles schémas
    ptr/.style={fill=gray!50,shape=ptr},
    chain2/.style={fill=RoyalBlue!50,shape=chain2}
}

\hypersetup{
  pdftitle={TP 3 - Gestion d'un dictionnaire arborescent}
}

%styles et formatage
\geometry{scale=0.8,centering}
\frenchbsetup{StandardLists=true}
\newcommand{\hsp}{\hspace{20pt}}
\titleformat{\chapter}[hang]{\LARGE\bfseries}{\thechapter\hsp\textcolor{lightgray}{|}\hsp}{0pt}{\LARGE\bfseries}

\newcommand{\NIL}{\texttt{\small NIL}}

\begin{document}
  \begin{titlepage}
  \newcommand{\HRule}{\rule{\linewidth}{0.5mm}}
  \center
  \null{}
  \vspace{3cm}

  \textsc{\LARGE ISIMA Première Année}\\[1.5cm]
  \textsc{\Large Compte-Rendu de TP}\\[0.5cm]
  \textsc{\LARGE Structures de Données}\\[1.5cm]
  \HRule \\[0.4cm]
  { \huge \bfseries Dérécursification à l'aide d'une pile}\\
  \HRule \\[1.5cm]

  \begin{minipage}{0.4\textwidth}
    \begin{flushleft} \large
      Benjamin BARBESANGE\\
      Pierre-Loup PISSAVY\\
      {\normalsize\textit{Groupe G21}}
    \end{flushleft}
  \end{minipage}
  ~
  \begin{minipage}{0.4\textwidth}
    \begin{flushright} \large
      \emph{Enseignant :} \\
      Michelle CHABROL
    \end{flushright}
  \end{minipage}\\[4cm]

  {\large mars 2015}\\[3cm]

  \vfill

  \includegraphics[width=6cm]{settings/ISIMA_logo.pdf}\\[1cm]
\end{titlepage}

  \tableofcontents
  \setlength{\parskip}{10pt}
  \setlength{\parindent}{0pt}
  \chapter{Présentation}
    Le but de ce TP est de créer une structure d'arbre permettant de gérer des mots d'un dictionnaire. Chaque liste des liens horizontaux est rangée par ordre alphabétique.

    Les opérations suivantes sont permises avec l'arbre:
    \begin{itemize}
      \item Creer l'arbre à partir de la notation parenthésée,
      \item Insérer un mot à la bonne place dans l'arbre,
      \item Afficher le contenu de l'arbre,
      \item Rechercher des mots commençant par un certain motif,
      \item Libérer la mémoire occupée par l'arbre.
    \end{itemize}

    \section{Structure de données employée}
      Les mots du dictionnaire sont rangés dans un arbre à liens horizontaux et verticaux, par orbre alphabétique en lecture préfixe. La fin d'un mot est signalée par une lettre majuscule.

      \begin{figure}[h]
  \begin{subfigure}{0.5\textwidth}
    \subcaption{Structure utilisée}
    \label{fig:struct}
    \begin{center}
    \shorthandoff{:}
    \begin{tikzpicture}[scale=0.5]
      \matrix at (0,0) {
        \node [case1] (a1) {max}; & \node [case1] (a11) {top}; & \node [case1,ptr] (a12) {val}; \\[1cm]
      };
      \matrix at (0, -4){
        \node [case2] (b1) {Valeurs dans la pile};\\
      };
      \draw [vers] (a12.south) |- +(0,-1) -| (b1.north);
    
    \end{tikzpicture}
    \shorthandon{:}
    \end{center}
  \end{subfigure}
  \begin{subfigure}{0.5\textwidth}
    \subcaption{Code}
    \label{fig:struct_code}
    \begin{center}
      \cfile[firstline=6,lastline=10]{../src/stack.h}
    \end{center}
  \end{subfigure}
  \caption{Structure et code correspondant}
\end{figure}


      Afin d'effectuer les tests, nous proposons une fonction basique effectuant des tests sommaires, ainsi que l'interprétation d'un fichier qui peut être donné comme premier paramètre.

      \newpage
      Le cas échéant, la structure de ce fichier doit respecter les règles suivantes:
      \begin{itemize}[noitemsep,nolistsep]
        \item 400 caractères au maximum par ligne,
        \item Les caractères suivants sont acceptés en début de ligne:
        \begin{description}[font=\texttt,noitemsep,nolistsep]
          \item [C :] Création d'arbre, doit contenir ensuite une représentation parenthésée,
          \item [I :] Insertion, peut contenir un mot ensuite,
          \item [M :] Recherche de motif, peut contenir un motif ensuite (chaîne de caractères),
          \item [L :] Libérer l'arbre,
          \item [A :] Afficher l'arbre,
          \item [\# :] Provoque l'affichage du texte qui suit (commentaire affiché).
        \end{description}
        \item Pour l'insertion, la casse n'a pas d'importance,
        \item Si l'on souhaite créer un nouvel arbre apres en avoir créé un premier, il est nécessaire de libérer ce dernier,
        \item Tout autre caractère ou bien une ligne vide provoqueront l'affichage d'une ligne vide.
      \end{itemize}
    \section{Organisation du code source}
      Nous avons découpé le TP en 3 parties. Une partie permet la gestion de pile, une autre la gestion de listes chaînées (qui sont utilisées dans la définition de l'arbre) et la dernière gère la structure d'arbre que nous avons créée.
    \subsection{Gestion de la pile}
  \begin{itemize}
	\item \bashinline{src/stack.h}
	\item \bashinline{src/stack.c}
  \end{itemize}

\subsection{Gestion des listes chaînées}
  \begin{itemize}
    \item \bashinline{src/list.h}
    \item \bashinline{src/list.c}
  \end{itemize}

\subsection{Gestion de l'arbre}
  \begin{itemize}
    \item \bashinline{src/tree.h}
    \item \bashinline{src/tree.c}
  \end{itemize}
  
\subsection{Programme principal}
  \begin{itemize}
    \item \bashinline{src/main.c}
  \end{itemize}


  \chapter{Détails du programme}
    \section{Gestion de liste chaînée}
  \cfile{../src/stack.h}
  \pagebreak
  \cfile{../src/stack.c}

\section{Gestion de news}
  \cfile{../src/truc.h}
  \cfile{../src/truc.c}

\pagebreak
\section{Programme principal}
  \cfile{../src/main.c}


  \chapter{Principes et lexiques des fonctions}
    Dans cette partie, sont décrits les algorithmes de principe associés aux fonctions écrites en langage C, ainsi qu'un lexique concernant les variables intermédiaires des fonctions.

Les lexiques des variables d'entrée, sortie et entrée/sortie sont disponibles dans le code source directement.

\section{Gestion de la pile}
  La gestion de la pile s'effectue grâce aux fichiers \texttt{stack.c} et \texttt{stack.h}.
  Les algorithmes de principe des différentes fonctions ont été précédement détaillés dans le \texttt{tp 2}, nous ne les détaillerons donc pas.

\section{Gestion des listes chaînées}
  Les fonctions de gestion des listes chaînées peuvent être trouvées dans les fichiers \texttt{list.c} et \texttt{list.h}.
  Les algorithmes de principe de ces fonctions ont également été fourni dans le \texttt{tp 1}. Ils ne seront donc pas inclus ici.

\section{Gestion de l'arbre}
  La gestion de l'arbre s'effectue avec les fonctions contenues dans \texttt{tree.c} et \texttt{tree.h}.
    
  \newpage
  \subsection{creerArbre}
    \begin{algo}[informal] %principe
      \ALGO{creerArbre (Principe)}
      \BEGIN
        \STATE{Initialise le code d'erreur à 0}
        \STATE{Initialise caractère cour, au début de la chaîne}
        \STATE{Initialise pointeur prec, de parcours à la racine}
        \STATE{Initialisation de la pile}
        \IF{l'initialisation de la pile est réussie}
          \STATE{Code d'erreur passe à 1}
          \WHILE{Code d'erreur \EQ 1 \AND (Pile non vide \OR caractere courant \NEQ ')')}
            \IF{cour \EQ ')'}
              \STATE{Push l'adresse du pointeur de parcours}
              \STATE{prec passe sur le lien vertical}
              \STATE{Avance d'un caractere dans la chaîne}{cour ++}
            \ELSE \IF{cour \EQ ','}
              \STATE{prec passe sur le lien horizontal}
              \STATE{Avance d'un caractère dans la chaîne}
            \ENDIF
            \STATE{On cree un noeud à l'adresse prec, avec le carctère courant}
            \IF{l'allocation a échouée}
              \STATE{Code d'erreur passe à 0}
            \ELSE
              \STATE{Avance d'un caractère dans la chaîne}
            \ENDIF
            \WHILE{Code d'erreur \EQ 1 \AND Pile non vide \AND cour \EQ ')')}
              \STATE{Pop dans un pointeur temporaire}
              \STATE{prec devient pointeur sur l'adresse du lien horizontal de ce que l'on vient de dépiler}
              \STATE{Avance d'un caractère dans la chaîne}
            \ENDWHILE
          \ENDWHILE
          \STATE{Libération de la pile}
        \ENDIF
        \RETURN{Code d'erreur}
      \END
    \end{algo}

    \begin{algo}[informal] %lexique
      \VAR
        \DECLVAR{p}{pile}
        \DECLVAR{**prec}{adresse du pointeur de parcours de l'arbre}
        \DECLVAR{*tmp}{pointeur temporaire lorsque l'on dépile}
        \DECLVAR{*cour}{caractère courant dans la chaîne}
        \DECLVAR{taille}{taille max de la pile (taille de la chaîne de catactère)}
        \DECLVAR{ret}{code d'erreur (1 si tout va bien, 0 sinon)}
      \ENDVAR
    \end{algo}

  \subsection{creerNoeud}
    \begin{algo}[informal] %principe
      \ALGO{creerNoeud (Principe)}
      \BEGIN
        \STATE{Allocation d'un nouvel élément}
        \IF{allocation réussie}
          \STATE{Le lien vertical de l'élément est NULL}
          \STATE{Le lien horizontal de l'élément est NULL}
          \STATE{La valeur de l'élément prend la valeur du paramètre}
        \ENDIF
        \RETURN{le nouvel élément crée}
      \END
    \end{algo}
    
    \begin{algo}[informal] %lexique
      \VAR
        \DECLVAR{*r}{nouvel élément crée}
      \ENDVAR
    \end{algo}

  \subsection{afficherArbrePref}
    \begin{algo}[informal] %principe
      \ALGO{afficherArbrePref (Principe)}
      \BEGIN
        \STATE{Initialisation de la pile}
        \STATE{Initialisation d'un pointeur cour, de parcours de l'arbre}
        \IF{cour \NEQ NULL \AND pile allouée}
          \REPEAT
            \WHILE{cour \NEQ NULL}
              \STATE{Push cour}
              \IF{la lettre dans cour est majuscule}{fin de mot}
                \STATE{Affiche le préfixe donné en paramètre}
                \STATE{Affiche le contenu de la pile}
                \STATE{Affiche un retour à la ligne}
              \ENDIF
            \ENDWHILE
            \WHILE{pile non vide \AND cour \EQ NULL}
              \STATE{Pop dans cour}
              \STATE{cour passe sur son lien horizontal}
            \ENDWHILE
          \ENDREPEAD[while]{pile non vilde \OR cour \NEQ NULL}
          \STATE{Libération de la pile}
        \ENDIF
      \END
    \end{algo}
    
    \begin{algo}[informal] %lexique
      \VAR
        \DECLVAR{p}{pile}
        \DECLVAR{*cour}{pointeur de parcours de l'arbre}
      \ENDVAR
    \end{algo}
  
  \subsection{afficherArbre}
  Ici, on appelle simplement la fonction précédente avec un préfixe valant la chaîne vide.
  
  \subsection{afficherPoint}
  Cette fonction affiche simplement la valeur d'un élément en convertissant le caractère en minuscule

  \subsection{insererMot}
    \begin{algo}[informal] %principe
      \ALGO{insererMot (Principe)}
      \BEGIN
        \RETURN{}
      \END
    \end{algo}

    \begin{algo}[informal] %lexique
      \VAR
        \DECLVAR{}{}
      \ENDVAR
    \end{algo}

  \subsection{libererArbre}
    \begin{algo}[informal] %principe
      \ALGO{libererArbre (Principe)}
      \BEGIN
        \RETURN{}
      \END
    \end{algo}

    \begin{algo}[informal] %lexique
      \VAR
        \DECLVAR{}{}
      \ENDVAR
    \end{algo}


  \chapter{Compte rendu d'exécution}
    \section{Makefile}
      \makefilefile{../src/Makefile}

    \section{Jeux de tests}
      Exécution du programme avec le fichier suivant:
\inputminted[frame=single,label=Test]{text}{../tests/test_insert}

\begin{figure}[H]
	\centering
	\includegraphics[width=18cm,clip=true,trim=1cm 4cm 5cm 12cm]{../tests/ddd_graph/chargement}
	\caption{Après lecture du fichier, contenu de \texttt{liste}}
\end{figure}

\begin{figure}[h!]
	\centering
	\includegraphics[width=15cm,clip=true,trim=1cm 4cm 5cm 14cm]{../tests/ddd_graph/nettoyage}
	\caption{Après suppression des messages obsolètes, contenu de \texttt{liste}}
\end{figure}

\begin{figure}[h!]
	\centering
	\includegraphics[width=15cm,clip=true,trim=1cm 4cm 5cm 14cm]{../tests/ddd_graph/modification}
	\caption{Après modification de date de début, contenu de \texttt{liste}}
\end{figure}

\begin{figure}[h!]
	\centering
	\includegraphics[width=2cm,clip=true,trim=1cm 4.5cm 25cm 15cm]{../tests/ddd_graph/suppression}
	\caption{Après suppression de la liste, contenu de \texttt{liste}}
\end{figure}

\begin{figure}[h!]
	\centering
	\inputminted[frame=single,label=Terminal]{text}{../tests/resultat_test_insert}
	\caption{Exécution du programme sur la sortie standard}
\end{figure}

\newpage

\begin{figure}[h!]
	\centering
	\inputminted[frame=single,label=Terminal]{text}{../tests/valgrind_report}
	\caption{Exécution avec valgrind}
\end{figure}

Voici un petit exemple de la recherche de motif sur le même fichier d'entrée. On recherche dans ce cas la chaîne \texttt{"liste"}, immédiatement après avoir chargé le fichier dans la liste.

\begin{figure}[h!]
	\centering
	\inputminted[frame=single,label=Terminal]{text}{../tests/resultat_test_motif}
	\caption{Exécution de la recherche de motif}
\end{figure}

\end{document}
