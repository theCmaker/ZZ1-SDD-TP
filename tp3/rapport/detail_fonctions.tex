Dans cette partie, sont décrits les algorithmes de principe associés aux fonctions écrites en langage C, ainsi qu'un lexique concernant les variables intermédiaires des fonctions.

Les lexiques des variables d'entrée, sortie et entrée/sortie sont disponibles dans le code source directement.

\section{Gestion de la pile}
  La gestion de la pile s'effectue grâce aux fichiers \texttt{stack.c} et \texttt{stack.h}.
  Les algorithmes de principe des différentes fonctions ont été précédement détaillés dans le \texttt{tp 2}, nous ne les détaillerons donc pas.

\section{Gestion des listes chaînées}
  Les fonctions de gestion des listes chaînées peuvent être trouvées dans les fichiers \texttt{list.c} et \texttt{list.h}.
  Les algorithmes de principe de ces fonctions ont également été fourni dans le \texttt{tp 1}. Ils ne seront donc pas inclus ici.

\section{Gestion de l'arbre}
  La gestion de l'arbre s'effectue avec les fonctions contenues dans \texttt{tree.c} et \texttt{tree.h}.
    
  \newpage
  \subsection{creerArbre}
    \begin{algo}[informal] %principe
      \ALGO{creerArbre (Principe)}
      \BEGIN
        \STATE{Initialise le code d'erreur à 0}
        \STATE{Initialise caractère cour, au début de la chaîne}
        \STATE{Initialise pointeur prec, de parcours à la racine}
        \STATE{Initialisation de la pile}
        \IF{l'initialisation de la pile est réussie}
          \STATE{Code d'erreur passe à 1}
          \WHILE{Code d'erreur \EQ 1 \AND (Pile non vide \OR caractere courant \NEQ ')')}
            \IF{cour \EQ ')'}
              \STATE{Push l'adresse du pointeur de parcours}
              \STATE{prec passe sur le lien vertical}
              \STATE{Avance d'un caractere dans la chaîne}{cour ++}
            \ELSE \IF{cour \EQ ','}
              \STATE{prec passe sur le lien horizontal}
              \STATE{Avance d'un caractère dans la chaîne}
            \ENDIF
            \STATE{On cree un noeud à l'adresse prec, avec le carctère courant}
            \IF{l'allocation a échouée}
              \STATE{Code d'erreur passe à 0}
            \ELSE
              \STATE{Avance d'un caractère dans la chaîne}
            \ENDIF
            \WHILE{Code d'erreur \EQ 1 \AND Pile non vide \AND cour \EQ ')')}
              \STATE{Pop dans un pointeur temporaire}
              \STATE{prec devient pointeur sur l'adresse du lien horizontal de ce que l'on vient de dépiler}
              \STATE{Avance d'un caractère dans la chaîne}
            \ENDWHILE
          \ENDWHILE
          \STATE{Libération de la pile}
        \ENDIF
        \RETURN{Code d'erreur}
      \END
    \end{algo}

    \begin{algo}[informal] %lexique
      \VAR
        \DECLVAR{p}{pile}
        \DECLVAR{**prec}{adresse du pointeur de parcours de l'arbre}
        \DECLVAR{*tmp}{pointeur temporaire lorsque l'on dépile}
        \DECLVAR{*cour}{caractère courant dans la chaîne}
        \DECLVAR{taille}{taille max de la pile (taille de la chaîne de catactère)}
        \DECLVAR{ret}{code d'erreur (1 si tout va bien, 0 sinon)}
      \ENDVAR
    \end{algo}

  \subsection{creerNoeud}
    \begin{algo}[informal] %principe
      \ALGO{creerNoeud (Principe)}
      \BEGIN
        \STATE{Allocation d'un nouvel élément}
        \IF{allocation réussie}
          \STATE{Le lien vertical de l'élément est NULL}
          \STATE{Le lien horizontal de l'élément est NULL}
          \STATE{La valeur de l'élément prend la valeur du paramètre}
        \ENDIF
        \RETURN{le nouvel élément crée}
      \END
    \end{algo}
    
    \begin{algo}[informal] %lexique
      \VAR
        \DECLVAR{*r}{nouvel élément crée}
      \ENDVAR
    \end{algo}

  \subsection{afficherArbrePref}
    \begin{algo}[informal] %principe
      \ALGO{afficherArbrePref (Principe)}
      \BEGIN
        \STATE{Initialisation de la pile}
        \STATE{Initialisation d'un pointeur cour, de parcours de l'arbre}
        \IF{cour \NEQ NULL \AND pile allouée}
          \REPEAT
            \WHILE{cour \NEQ NULL}
              \STATE{Push cour}
              \IF{la lettre dans cour est majuscule}{fin de mot}
                \STATE{Affiche le préfixe donné en paramètre}
                \STATE{Affiche le contenu de la pile}
                \STATE{Affiche un retour à la ligne}
              \ENDIF
            \ENDWHILE
            \WHILE{pile non vide \AND cour \EQ NULL}
              \STATE{Pop dans cour}
              \STATE{cour passe sur son lien horizontal}
            \ENDWHILE
          \ENDREPEAD[while]{pile non vilde \OR cour \NEQ NULL}
          \STATE{Libération de la pile}
        \ENDIF
      \END
    \end{algo}
    
    \begin{algo}[informal] %lexique
      \VAR
        \DECLVAR{p}{pile}
        \DECLVAR{*cour}{pointeur de parcours de l'arbre}
      \ENDVAR
    \end{algo}
  
  \subsection{afficherArbre}
  Ici, on appelle simplement la fonction précédente avec un préfixe valant la chaîne vide.
  
  \subsection{afficherPoint}
  Cette fonction affiche simplement la valeur d'un élément en convertissant le caractère en minuscule

  \subsection{insererMot}
    \begin{algo}[informal] %principe
      \ALGO{insererMot (Principe)}
      \BEGIN
        \RETURN{}
      \END
    \end{algo}

    \begin{algo}[informal] %lexique
      \VAR
        \DECLVAR{}{}
      \ENDVAR
    \end{algo}

  \subsection{libererArbre}
    \begin{algo}[informal] %principe
      \ALGO{libererArbre (Principe)}
      \BEGIN
        \RETURN{}
      \END
    \end{algo}

    \begin{algo}[informal] %lexique
      \VAR
        \DECLVAR{}{}
      \ENDVAR
    \end{algo}
