Dans cette partie, sont décrits les algorithmes de principe associés aux fonctions écrites en langage C, ainsi qu'un lexique concernant les variables intermédiaires des fonctions.

Les lexiques des variables d'entrée, sortie et entrée/sortie sont disponibles dans le code source directement.

\section{Gestion de la pile}
  La gestion de la pile s'effectue grâce aux fichiers \texttt{stack.c} et \texttt{stack.h}.
  Les algorithmes de principe des différentes fonctions ont été précédement détaillés dans le \texttt{tp 2}, nous ne les détaillerons donc pas.

\section{Gestion des listes chaînées}
  Les fonctions de gestion des listes chaînées peuvent être trouvées dans les fichiers \texttt{list.c} et \texttt{list.h}.
  Les algorithmes de principe de ces fonctions ont également été fourni dans le \texttt{tp 1}. Ils ne seront donc pas inclus ici.

\section{Gestion de l'arbre}
  La gestion de l'arbre s'effectue avec les fonctions contenues dans \texttt{tree.c} et \texttt{tree.h}.
    
  \newpage
  \subsection{creerArbre}
    \begin{algo}[informal] %principe
      \ALGO{creerArbre (Principe)}
      \BEGIN
        \STATE{Initialise le code d'erreur à 0}
        \STATE{Initialise caractère cour, au début de la chaîne}
        \STATE{Initialise pointeur prec, de parcours à la racine}
        \STATE{Initialisation de la pile}
        \IF{l'initialisation de la pile est réussie}
          \STATE{Code d'erreur passe à 1}
          \WHILE{Code d'erreur \EQ 1 \AND (Pile non vide \OR caractere courant \NEQ ')')}
            \IF{cour \EQ ')'}
              \STATE{Push l'adresse du pointeur de parcours}
              \STATE{prec passe sur le lien vertical}
              \STATE{Avance d'un caractere dans la chaîne}{cour ++}
            \ELSE \IF{cour \EQ ','}
              \STATE{prec passe sur le lien horizontal}
              \STATE{Avance d'un caractère dans la chaîne}
            \ENDIF
            \STATE{On cree un noeud à l'adresse prec, avec le carctère courant}
            \IF{l'allocation a échouée}
              \STATE{Code d'erreur passe à 0}
            \ELSE
              \STATE{Avance d'un caractère dans la chaîne}
            \ENDIF
            \WHILE{Code d'erreur \EQ 1 \AND Pile non vide \AND cour \EQ ')')}
              \STATE{Pop dans un pointeur temporaire}
              \STATE{prec devient pointeur sur l'adresse du lien horizontal de ce que l'on vient de dépiler}
              \STATE{Avance d'un caractère dans la chaîne}
            \ENDWHILE
          \ENDWHILE
          \STATE{Libération de la pile}
        \ENDIF
        \RETURN{Code d'erreur}
      \END
    \end{algo}

    \begin{algo}[informal] %lexique
      \VAR
        \DECLVAR{p}{pile}
        \DECLVAR{**prec}{adresse du pointeur de parcours de l'arbre}
        \DECLVAR{*tmp}{pointeur temporaire lorsque l'on dépile}
        \DECLVAR{*cour}{caractère courant dans la chaîne}
        \DECLVAR{taille}{taille max de la pile (taille de la chaîne de catactère)}
        \DECLVAR{ret}{code d'erreur (1 si tout va bien, 0 sinon)}
      \ENDVAR
    \end{algo}

  \subsection{creerNoeud}
    \begin{algo}[informal] %principe
      \ALGO{creerNoeud (Principe)}
      \BEGIN
        \STATE{Allocation d'un nouvel élément}
        \IF{allocation réussie}
          \STATE{Le lien vertical de l'élément est NULL}
          \STATE{Le lien horizontal de l'élément est NULL}
          \STATE{La valeur de l'élément prend la valeur du paramètre}
        \ENDIF
        \RETURN{le nouvel élément crée}
      \END
    \end{algo}
    
    \begin{algo}[informal] %lexique
      \VAR
        \DECLVAR{*r}{nouvel élément crée}
      \ENDVAR
    \end{algo}

  \subsection{afficherArbrePref}
    \begin{algo}[informal] %principe
      \ALGO{afficherArbrePref (Principe)}
      \BEGIN
        \STATE{Initialisation de la pile}
        \STATE{Initialisation d'un pointeur cour, de parcours de l'arbre}
        \IF{cour \NEQ NULL \AND pile allouée}
          \REPEAT
            \WHILE{cour \NEQ NULL}
              \STATE{Push cour}
              \IF{la lettre dans cour est majuscule}{fin de mot}
                \STATE{Affiche le préfixe donné en paramètre}
                \STATE{Affiche le contenu de la pile}
                \STATE{Affiche un retour à la ligne}
              \ENDIF
            \ENDWHILE
            \WHILE{pile non vide \AND cour \EQ NULL}
              \STATE{Pop dans cour}
              \STATE{cour passe sur son lien horizontal}
            \ENDWHILE
          \ENDREPEAD[while]{pile non vilde \OR cour \NEQ NULL}
          \STATE{Libération de la pile}
        \ENDIF
      \END
    \end{algo}
    
    \begin{algo}[informal] %lexique
      \VAR
        \DECLVAR{p}{pile}
        \DECLVAR{*cour}{pointeur de parcours de l'arbre}
      \ENDVAR
    \end{algo}
  
  \subsection{afficherArbre}
  Ici, on appelle simplement la fonction précédente avec un préfixe valant la chaîne vide.
  
  \subsection{afficherPoint}
  Cette fonction affiche simplement la valeur d'un élément en convertissant le caractère en minuscule

  \subsection{libererArbre}
    \begin{algo}[informal] %principe
      \ALGO{libererArbre (Principe)}
      \BEGIN
        \STATE{Initialisation d'une pile}
        \STATE{Initialisation d'un pointeur cour, sur la tête de l'arbre}
        \IF{cour \NEQ NULL \AND Pile initialisée}
          \REPEAT
            \WHILE{cour \NEQ NULL}
              \STATE{Place la valeur cour dans un pointeur temporaire}
              \IF{Lien horizontal de cour \NEQ NULL}
                \STATE{Push l'adresse dans la pile}{Sauvegarde pour y revenir}
              \ENDIF
              \STATE{Passe au lien vertical de cour}
              \STATE{Libération de l'élément contenu dans le pointeur temporaire}
            \ENDWHILE
            \IF{Pile non vide}{Il reste des éléments à libérer}
              \STATE{Pop la pile dans cour}
            \ENDIF
          \ENDREPEAD[while]{Pile non vide \OR cour \NEQ NULL}
          \STATE{Libération de la pile}
        \ENDIF
        \STATE{Mise du pointeur de tête de l'arbre sur NULL}
      \END
    \end{algo}

    \begin{algo}[informal] %lexique
      \VAR
        \DECLVAR{p}{pile}
        \DECLVAR{*cour}{pointeur de parcours de l'arbre}
        \DECLVAR{*tmp}{pointeur temporaire servant à libérer les éléments}
      \ENDVAR
    \end{algo}
  
  \subsection{adj\_fils}
  Cette fonction permet simplement d'ajouter un fils à un élément. Le pointeur sur lien vertical de l'élément prend l'adresse du précédent et le pointeur prec prend l'adresse de l'élément que l'on ajoute.
  
  \subsection{rech\_mot}
    \begin{algo}[informal] %principe
      \ALGO{rech\_mot (Principe)}
      \BEGIN
        \STATE{Initialisation d'un pointeur cour, sur le caractère courant du mot}
        \STATE{Initialisation d'un pointeur arbre, de parcours de l'arbre}
        \STATE{Initialisation d'un booleen existe}{Initialisé à TRUE}
        \WHILE{existe \EQ TRUE \AND arbre \NEQ NULL \AND cour est en minuscule}{Recherche du début du mot}
          \STATE{Lance rech\_prec() et stocke les résultats dans arbre et existe}
          \IF{existe \EQ TRUE}
            \STATE{On passe arbre sur l'adresse de son fils}
            \STATE{Avance d'un caractère dans le mot}
          \ENDIF
        \ENDWHILE
        \IF{arbre \NEQ NULL \AND cour en majuscule}
          \STATE{Lance rech\_prec() et stocke les résultats dans arbre et existe}
          \IF{la valeur du noeud courant \EQ cour}
            \STATE{Avance d'un caractère dans le mot}
          \ENDIF
        \ENDIF
        \STATE{Le mot en paramètre prend la valeur de cour}{Ne contiendra que les lettres du mot non traitées}
        \RETURN{arbre}
      \END
    \end{algo}
    
    \begin{algo}[informal] %lexique
      \VAR
        \DECLVAR{*cour}{pointeur de parcours du mot en paramètre}
        \DECLVAR{**arbre}{double pointeur de parcours de l'arbre}
        \DECLVAR{existe}{booleen d'existence du caractère courant du mot dans l'arbre}
      \ENDVAR
    \end{algo}
    
  \subsection{insererMot}
    \begin{algo}[informal] %principe
      \ALGO{insererMot (Principe)}
      \BEGIN
        \STATE{Initialisation d'un pointeur arbre à la tête}
        \STATE{Initialisation d'un code d'erreur à 1}
        \IF{le mot en entré \NEQ '\0'}{Le mot n'est pas vide}
          \STATE{On cree une copie du mot entré}
          \STATE{On passe chaque lettre de cette copie en minuscule, sauf la dernière}{Avec une simple boucle while}
          \STATE{Lance rech\_mot() et stocke les résultats dans arbre et cour}
          \IF{cour \NEQ '\0'}{Le mot n'est pas présent}
            \IF{arbre \NEQ NULL \AND valeur de l'élémént arbre \EQ cour(en minuscule)}
              \STATE{On passe la lettre dans arbre en majuscule}
              \STATE{Avance d'un caractère dans le mot}
            \ELSE
              \STATE{Creation d'un nouveau noeud avec cour comme valeur}
              \IF{élément correctement crée}
                \STATE{Ajout de l'élément dans le lien horizontal de arbre}
                \STATE{On passe arbre sur l'adresse du pointeur de son fils}
                \STATE{Avance d'un caractère dans le mot}
                \WHILE{code d'erreur \NEQ 0 \AND cour \NEQ '\0'}{Insère les lettres restantes}
                  \STATE{Creation d'un nouveau noeud avec cour comme valeur}
                  \IF{élément correctement crée}
                    \STATE{Ajout de l'élément dans le lien vertical de arbre}
                    \STATE{On passe arbre sur l'adresse du pointeur de son fils}
                    \STATE{On avance d'un caractère dans le mot}
                  \ELSE
                    \STATE{Code d'erreur passe à 0}{Problème d'allocation}
                  \ENDIF
                \ENDWHILE
              \ELSE
                \STATE{Code d'erreur passe à 0}{Problème d'allocation}
              \ENDIF
            \ENDIF
          \ENDIF
        \ENDIF
        \RETURN{Code d'erreur}
      \END
    \end{algo}
    
    \begin{algo}[informal] %lexique
      \VAR
        \DECLVAR{len}{taille du mot en paramètre}
        \DECLVAR{i}{indice de boucle pour copie du mot}
        \DECLVAR{res}{code d'erreur (1 si tout s'est bien passé, 0 sinon)}
        \DECLVAR{*cour}{poiteur sur caractère courant du mot (copié)}
        \DECLVAR{*tmp}{pointeur temporaire pour la création de nouveaux noeuds}
        \DECLVAR{**arbre}{pointeur de parcours de l'arbre}
      \ENDVAR
    \end{algo}
    
  \subsection{rech\_motif}
    \begin{algo}[informal] %principe
      \ALGO{rech\_motif (Principe)}
      \BEGIN
        \STATE{Initialisation d'un pointeur arbre, sur la tête}
        \STATE{Initialisation d'un pointeur cour, sur la caractère courant du mot}
        \STATE{Lance rech\_mot() et stocke les résultats dans arbre et cour}
        \IF{cour \EQ '\0'}
          \STATE{Affichage de l'arbre avec pour préfixe le motif entré}
        \ENDIF
      \END
    \end{algo}
  
    \begin{algo}[informal] %lexique
      \VAR
        \DECLVAR{**arbre}{pointeur de parcours de l'arbre}
        \DECLVAR{*cour}{pointeur sur le caractère courant de l'arbre}
      \ENDVAR
    \end{algo}
