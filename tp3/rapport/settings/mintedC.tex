\usepackage{upquote}
\usepackage{textcomp}
\usepackage{minted}
\usepackage{inconsolata}
\renewcommand{\listingscaption}{Code}
\usepackage{MnSymbol}
\newminted{c}{
  linenos                = true,
  breaklines             = true,
  frame                  = single,
  breakautoindent        = true,
  breaksymbolleft        = $\lhookrightarrow$,
  breaksymbolindentleft  = 10pt,
  breaksymbolsepleft     = 2pt,
  breaksymbolright       = $\rhookleftarrow$,
  breaksymbolindentright = 10pt,
  breaksymbolsepright    = 2pt,
  label                  = Code C,
  texcomments            = true
}

\newmintedfile{c}{
  linenos                = true,
  breaklines             = true,
  frame                  = single,
  breakautoindent        = true,
  breaksymbolleft        = $\lhookrightarrow$,
  breaksymbolindentleft  = 10pt,
  breaksymbolsepleft     = 2pt,
  breaksymbolright       = $\rhookleftarrow$,
  breaksymbolindentright = 10pt,
  breaksymbolsepright    = 2pt,
  label                  = Code C,
  texcomments            = false
}

\newmintinline{c}{
}

% --------------------------------------------- 
% --------------- How to use it --------------- 
% --------------------------------------------- 
%                                               
% Specific environment: ccode and ccode*
% ccode  does not accept options
% ccode* needs an argument that contains the
%        required options
%
% \begin{ccode*}{gobble=2}
%   #include <stdio.h>
%   #include "fun.h"
% 
%   #define N 15
% 
%   typedef struct alpha {
%     int           number;
%     char          letter;
%     struct alpha *next;
%   } alpha_t;
% 
%   int main (int argc, char **argv) {
%     int i = 0;
%     for (i = 0; i < N; ++i) {
%       printf("Hello World! %d\n",10);
%     } 
%     return EXIT_SUCCESS; /* Comment */
%   }//TODO: next comment is a very long one, so long that it's reaching the right side of the frame. \Smiley{}
% 
%   void display (alpha_t x) {
%     printf("%d, %c",x.number, x.letter);
%   }
% \end{ccode*}
%
% ---------------------------------------------
% Inline use: among the text: cinline
%
% \cinline{printf}
%
% ---------------------------------------------
% Highlighting source code from file: cfile
%
% \cfile{main.c}
%
% ---------------------------------------------
% listing environment, the same as figure, with
% captions etc.
%
% \begin{listing}[H]
%   %Your code here
%   \caption{Example of a listing.}
%   \label{lst:example}
% \end{listing}
%
% ---------------------------------------------
% ---------------------------------------------