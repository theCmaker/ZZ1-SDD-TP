\begin{figure}[H]
    \begin{center}
      \shorthandoff{:}
      \begin{tikzpicture}[>=latex]
        \node [ptr,label=below:r] (A0) at (0,0) {};
        \node [chain2] (A1) at (2,0) {a};
        \node [chain2] (A2) at (2,-1.5) {m};
        \node [chain2] (A3) at (8,0) {r};
        \node [chain2] (A4) at (2,-3) {a};
        \node [chain2] (A5) at (5,-3) {I};
        \node [chain2] (A6) at (8,-1.5) {a};
        \node [chain2] (A7) at (11,-1.5) {i};
        \node [chain2] (A8) at (2,-4.5) {S};
        \node [chain2] (A12) at (8,-3) {T};
        \node [chain2] (A14) at (11,-3) {Z};
        \node [chain2] (A10) at (5,-4.5) {S};
        \foreach \m in {8,12,10,14} {
          \node at (A\m.ptr1) {\NIL};
        }
        \foreach \m in {2,3,5,7,8,12,10,14} {
          \node at (A\m.ptr2) {\NIL};
        }
        \foreach \m in {1,...,7} {
          \pgfmathtruncatemacro\p{2*\m}
          \draw [->] (A\m.ptr1) -- (A\p.north);
        }
        \foreach \m in {0,4,6} {
          \pgfmathtruncatemacro\p{\m + 1}
          \draw [->] (A\m.next) -- (A\p.west);
        }
        \draw [->] (A1.ptr2) -- (A3.west);
      \end{tikzpicture}
      \shorthandon{:}
    \end{center}
    \caption{Représentation en liens vertical et horizontal}
    \label{fig:lvlh}
  \end{figure}